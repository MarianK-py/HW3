% !TeX spellcheck = sk_SK-Slovak
\documentclass[a4paper]{article}
\usepackage[slovak]{babel}
\usepackage[utf8]{inputenc}
\usepackage[T1]{fontenc}
\usepackage{a4wide}
\usepackage{amsmath}
\usepackage{amsfonts}
\usepackage{amsthm,amssymb}
\usepackage{mathrsfs}
\usepackage[small,bf]{caption}
\usepackage{subcaption}
\usepackage{xcolor}
\usepackage{graphicx}
\usepackage{enumerate}
\usepackage{hyperref}
\usepackage[a4paper, total={7in, 10.2in}]{geometry}



\pagestyle{empty}
\setlength{\parindent}{0pt}

\newenvironment{modenumerate}
{\enumerate\setupmodenumerate}
{\endenumerate}

%\renewcommand{\thesubsection}{\thesection.\alph{subsection}}
\renewcommand{\thesubsection}{\alph{subsection})}


\begin{document} 
	
	\pagenumbering{arabic}
	\pagestyle{plain}
	
	\begin{center}
		\sc\large
		PHYSICAL BASED ANIMATIONS AND MATHEMATICAL MODELING HW 2 
		\\
		Matica hybnosti
	\end{center}

	Autor: Marián Kravec
	\\
	\\
	Našou úlohou je vypočítať hmotnosť, tensor zotrvačnosti a ťažisko následujúceho telesa:
	
	\centerline{\includegraphics[width=0.7\textwidth]{podorysy}} 
	
	Prvá vec čo si môžeme všimnúť je, že aj napriek tomu, že naše teleso vyzerá pomerne komplikovane vieme ho rozdeliť na trojicu hranolov z ktorých dve hranoly sú dokonca kocky.
	\\
	\\
	To nám úlohu zjednodušuje keďže vieme, že pre hranol sa matica zotrvačnosti vypočíta následovne:
	\begin{align*}
		J_0 = \begin{bmatrix}
			\frac{m}{12}(h^2 + d^2) & 0 & 0 \\
			0 & \frac{m}{12}(w^2 + d^2) & 0 \\
			0 & 0 & \frac{m}{12}(w^2 + h^2)
		\end{bmatrix}
	\end{align*}
	Kde $(w, h, d)$ sú rozmery hranolu a $m$ je jeho hmotnosť. Avšak tento vzorec platí iba ak os otáčania prechádza ťažiskom daného hranolu. Ak je tento hranol posunutý v súradnicovej sústave o vektor $\boldsymbol{r}$ tak jeho maticu hybnosti vieme vypočítať následovne:
	\begin{align*}
		J = J_0 + m(r^TrI-rr^T)
	\end{align*}
	Poďme si teraz vypočítať rozmery jednotlivých našich rozmerov. Ak sa nemýlim v zadaní nie je ich poloha a rozmery určené presnejšie ako z pohľadu na ich vizualizáciu čiže pôdorysy. Z pôdorysov vidíme, že všetky 3 hranoly majú všetky hrany rovnobežné s niektorou zo štandardných osí čo uľahčuje určovanie ich rozmerov.
	\\
	\\
	Začnime s kockou ktorá je na pôdorysoch zobrazená ako najtmavšia, označíme si ju ako kváder $k_1$. Vidíme, že tento hranol má všetky rozmery 2 metre, ak budeme považovať meter za základnú jednotku nášho priestoru tak rozmery vieme zapísať ako $(w_1, h_1, d_1) = (2, 2, 2)$. 
	\\
	\\
	Ak sa pozrieme na kocku na opačnej strana telesa (najsvetlejšia)(označíme ako kváder $k_2$) vidíme, že jej rozmery sú totožné čiže takisto ju vieme zapísať $(w_2, h_2, d_2) = (2, 2, 2)$.
	\\
	\\
	Nakoniec nám zostal hranol v strede ktorý označíme ako kváder $k_3$. Tu si z prvého pôdorysu môžeme všimnúť, že v smere osi $x$ má náš kváder šírku $3$. Zároveň v smere osi $y$ vidíme, že má výšku $1$, hodnotu hĺbky $z$ vidíme na druhom a treťom pôdoryse a je to hodnota $1$. Takže náš kváder vieme zapísať ako $(w_3, h_3, d_3) = (3, 1, 1)$.
	\\
	\\
	Predtým, než sa pustíme do ďalších výpočtov ešte si zadefinujme hustotu tohto objektu. Keďže som sa narodil 18.9. tak v mojom prípade $x=1$ a $y=8$, čiže ak hustota je $\rho=1.xy\frac{kg}{m^3}$ tak v mojom prípade je to $\rho=1.18\frac{kg}{m^3}$.
	\\
	\newpage
	\subsection{}
	
	Ako prvé vypočítame hmotnosť celého telesa. Vieme, že celková hmotnosť telesa $m$ je súčet hmotností jeho častí. Preto ju vieme zapísať ako $m=\sum_{i=1}^{3}m_i$ kde $m_i$ je hmotnosť kvádra $k_i$.
	\\
	O hmotnosti kvádra vieme, že ju vieme vypočítať ako objem kvádra $V$ vynásobený hustotou $\rho$ ($m = V \cdot \rho$). 
	\\
	Objem kvádra vieme vypočítať ako súčin jeho rozmerov, čiže $V=w \cdot h \cdot d$
	\\
	Takže ďalším krokom bude výpočet objemu $V_i$ našich kvádrov (rozmery kvádrov sme počítali v metroch takže je to priamočiare). 
	\begin{align*}
		&V_1 = w_1 \cdot h_1 \cdot d_1 = 2 \cdot 2 \cdot 2 = 8 m^3 
		\\
		&V_2 = w_2 \cdot h_2 \cdot d_2 = 2 \cdot 2 \cdot 2 = 8 m^3 
		\\
		&V_3 = w_3 \cdot h_3 \cdot d_3 = 3 \cdot 1 \cdot 1 = 3 m^3
		\\
		&\text{Ako ďalšie vypočítame hmotnosti jednotlivých kvádrov}
		\\
		&\text{(všetky kvádre majú rovnakú hustotu $\rho$)} 
		\\
		&m_1 = V_1 \cdot \rho = 8 \cdot 1.18 = 9.44 kg
		\\
		&m_2 = V_2 \cdot \rho = 8 \cdot 1.18 = 9.44 kg
		\\
		&m_3 = V_3 \cdot \rho = 3 \cdot 1.18 = 3.54 kg
		\\
		&\text{Teraz môžeme vypočítať celkovú hmotnosť objektu}
		\\
		&m = \sum_{i=1}^{3} m_i = m_1 + m_2 + m_3 = 9.44 + 9.44 + 3.54 = 22.42 kg
	\end{align*}  
	Takže celková hmotnosť telesa je $m = 22.42 kg$.
	
	\subsection{}
	
	Ďalej chcem vypočítať tenzor zotrvačnosti $J$ ktorý vieme vypočítať ako súčet tenzorov jednotlivých častí objektu $J = \sum_{i=1}^{3} J_i$.
\end{document}
